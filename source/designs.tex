\documentclass[aspectratio=1610]{beamer}
\usepackage[T1]{fontenc}
\usetheme{wildcat}


\title{Wildcat Beamer Theme}
\date{January 2024}
\author{Aaron Wolf (Northwestern University)}


\begin{document}




% Facet (Default)
\setbeamertemplate{background}{
    \begin{tikzpicture}
        \useasboundingbox (0,0) rectangle(\the\paperwidth,\the\paperheight);
        \draw[color=wcprimary,fill=wcprimary, line width=0.05pt] (0,0) -- (0,0.9*\the\paperheight) -- (4.4,0.75*\the\paperheight) -- (6.4,0) -- cycle;
        \draw[color=wcprimary110,fill=wcprimary110,, line width=0.05pt] (0,\the\paperheight) -- (0,0.9*\the\paperheight) -- (4.4,0.75*\the\paperheight) -- (6.4,\the\paperheight) -- cycle;
        \draw[color=wcprimary120,fill=wcprimary120, line width=0.05pt] (6.4,\the\paperheight) -- (4.4,0.75*\the\paperheight) -- (6.4,0) -- (7.7,0) -- (13.2,0.82*\the\paperheight) -- (12.8,\the\paperheight) -- cycle;
        \draw[color=wcprimary130,fill=wcprimary130, line width=0.05pt] (7.7,0) -- (13.2,0.82*\the\paperheight) -- (11.6,0) -- cycle;
        \draw[color=wcprimary110,fill=wcprimary110, line width=0.05pt] (11.6,0) -- (13.2,0.82*\the\paperheight) -- (\the\paperwidth,0.55*\the\paperheight) -- (\the\paperwidth,0.18*\the\paperheight) -- (12,0) -- cycle;
        \draw[color=wcprimary,fill=wcprimary, line width=0.05pt] (12,0) -- (\the\paperwidth,0.18*\the\paperheight) -- (\the\paperwidth,0) -- cycle;
        \draw[color=wcprimary,fill=wcprimary,  line width=0.05pt] (12.8,\the\paperheight) -- (13.2,0.82*\the\paperheight) -- (\the\paperwidth,0.55*\the\paperheight) -- (\the\paperwidth,\the\paperheight) -- cycle;        
    \end{tikzpicture}
    }
    \setbeamertemplate{footline}{}
\begin{frame}{Facet (Default)}
\end{frame}

% Plain
\setbeamertemplate{background}{
    \begin{tikzpicture}
        \useasboundingbox (0,0) rectangle(\the\paperwidth,\the\paperheight);
        \fill[fill=wcprimary] (0,0) rectangle(\the\paperwidth,\the\paperheight);
    \end{tikzpicture}
    }
    \setbeamertemplate{footline}{}
\begin{frame}{Plain}
\end{frame}

%%%%%%%%%%%%%%%%%%%%%%%%%%%%%%%%%%%%%%%%%%%%%%%%%%%%%%%%%%%%%%%
% Geometric Designs
%%%%%%%%%%%%%%%%%%%%%%%%%%%%%%%%%%%%%%%%%%%%%%%%%%%%%%%%%%%%%%%

% Facet (2)
\setbeamertemplate{background}{
    \begin{tikzpicture}
        \useasboundingbox (0,0) rectangle(\the\paperwidth,\the\paperheight);
        \fill[fill=wcprimary] (0,0) rectangle(\the\paperwidth,\the\paperheight);
        \pgfmathsetseed{263}
        % Create 3 nodes at the top of the page, middle, and bottom
        % Bottom
        \pgfmathrandominteger{\xBA}{-2}{4}
        \pgfmathrandominteger{\xBB}{5}{8}
        \pgfmathrandominteger{\xBC}{9}{20}

        % Middle
        \pgfmathrandominteger{\xMB}{3}{13}
        \pgfmathrandominteger{\yMA}{3}{10}
        \pgfmathrandominteger{\yMB}{1}{\yMA-1}
        \pgfmathrandominteger{\yMC}{2}{10}

        % Top
        \pgfmathrandominteger{\xTA}{-2}{4}
        \pgfmathrandominteger{\xTB}{5}{8}
        \pgfmathrandominteger{\xTC}{9}{20}

        % Colors
        \pgfmathrandominteger{\cA}{96}{104}
        \pgfmathrandominteger{\cB}{96}{104}
        \pgfmathrandominteger{\cC}{96}{104}
        \pgfmathrandominteger{\cD}{96}{104}
        \pgfmathrandominteger{\cE}{96}{104}
        \pgfmathrandominteger{\cF}{96}{104}
        \pgfmathrandominteger{\cG}{96}{104}
        \pgfmathrandominteger{\cH}{96}{104}
        

        % Draw the polygons by connecting the nodes
        \filldraw[fill=wcprimary!\cA,draw=wcprimary!\cA,line width = 0.05pt] (\xBA,0) -- (\xBC,0) -- (\xBB*0.75,\yMB) --  (0,\yMA) -- cycle;
        \filldraw[fill=wcprimary!\cB,draw=wcprimary!\cB,line width = 0.05pt] (\xBB*0.75,\yMB) --  (0,\yMA) -- (0.75\xBA,\the\paperheight) -- (0.9*\xBB,\the\paperheight) -- cycle;
        \filldraw[fill=wcprimary!\cC,draw=wcprimary!\cC,line width = 0.05pt] (0.9*\xBB,\the\paperheight) -- (\xBB*0.75,\yMB) -- (\xBC,0) -- cycle;
        \filldraw[fill=wcprimary!\cD,draw=wcprimary!\cD,line width = 0.05pt] (\xBC,0) -- (1.5*\xBC,\yMC) -- (1.2*\xBC,\the\paperheight) -- (0.9*\xBB,\the\paperheight) -- cycle;
        \filldraw[fill=wcprimary!\cE,draw=wcprimary!\cE,line width = 0.05pt] (1.2*\xBC,\the\paperheight) -- (1.5*\xBC,\yMC) -- (\the\paperwidth,\the\paperheight) -- cycle;
        \filldraw[fill=wcprimary!\cF,draw=wcprimary!\cF,line width = 0.05pt] (1.5*\xBC,\yMC) -- (\the\paperwidth,\the\paperheight) -- (\the\paperwidth,0) -- cycle;
        \filldraw[fill=wcprimary!\cG,draw=wcprimary!\cG,line width = 0.05pt] (1.5*\xBC,\yMC) -- (\xBC,0) -- (\the\paperwidth,0) -- cycle;
        \filldraw[fill=wcprimary!\cH,draw=wcprimary!\cH,line width = 0.05pt] (\xBB*0.5,0) -- (\xBC,0) -- (\xBB*0.75,\yMB) -- cycle;
    \end{tikzpicture}
    }
    \setbeamertemplate{footline}{}
\begin{frame}{Facet (2)}
\end{frame}


% Triangles (1)
\newcommand{\drawTriangle}[6]{
    \filldraw[fill=wcprimary!#3,draw=wcprimary!#3,line width = 0.05pt] (#1-1,#2-1) -- (#1-0.5,#2-0.5) -- (#1,#2-1) -- cycle;
    \filldraw[fill=wcprimary!#4,draw=wcprimary!#4,line width = 0.05pt] (#1-1,#2-1) -- (#1-0.5,#2-0.5) -- (#1-1,#2) -- cycle;
    \filldraw[fill=wcprimary!#5,draw=wcprimary!#5,line width = 0.05pt] (#1-1,#2) -- (#1-0.5,#2-0.5) -- (#1,#2) -- cycle;
    \filldraw[fill=wcprimary!#6,draw=wcprimary!#6,line width = 0.05pt] (#1,#2) -- (#1-0.5,#2-0.5) -- (#1,#2-1) -- cycle;
}

\setbeamertemplate{background}{
    \begin{tikzpicture}
        \useasboundingbox (0,0) rectangle(\the\paperwidth,\the\paperheight);
        %\drawTriangle{1}{1}{}
        \pgfmathsetseed{6523}
        \foreach \x in {1,...,20} {
            \foreach \y in {1,...,16} {
                \pgfmathrandominteger{\zA}{95}{105}
                \pgfmathrandominteger{\zB}{95}{105}
                \pgfmathrandominteger{\zC}{95}{105}
                \pgfmathrandominteger{\zD}{95}{105}
                \drawTriangle{\x}{\y}{\zA}{\zB}{\zC}{\zD}
            }
        }
        \end{tikzpicture}
    }
    \setbeamertemplate{footline}{}
\begin{frame}{Triangles (1)}
\end{frame}

% Triangles (2)
\setbeamertemplate{background}{
    \begin{tikzpicture}[
        x=0.1*\the\paperheight,
        y=0.1*\the\paperheight,
        square/.style={draw,fill,minimum size=1.5pt,inner sep=0pt},
        dot/.style={circle,circle,fill,minimum size=2pt,inner sep=0pt}
        ]
        \useasboundingbox (0,0) rectangle(\the\paperwidth,\the\paperheight);
        \fill[fill=wcprimary] (0,0) rectangle(\the\paperwidth,\the\paperheight);
        \pgfmathsetseed{5984}
        \pgfmathsetmacro{\height}{sqrt(3)/2}
        % draw triangular grid
        \foreach \x in {0,...,20} {
            \foreach \y in {0,...,12} {
                \pgfmathrandominteger{\colorboxl}{95}{105}
                \pgfmathrandominteger{\colorboxr}{95}{105}
                \coordinate (A) at (\x,\y*\height);
                \coordinate (B) at (\x+1,\y*\height);
                \coordinate (C) at (\x+0.5,\y*\height+\height);
                \coordinate (D) at (\x-0.5,\y*\height+\height);
                \filldraw[fill=wcprimary!\colorboxl,draw=wcprimary!\colorboxl,line width=0.05pt] (A) -- (B) -- (C) -- cycle;
                \filldraw[fill=wcprimary!\colorboxr,draw=wcprimary!\colorboxr,line width=0.05pt] (A) -- (C) -- (D) -- cycle;
            }
        }
    \end{tikzpicture}
            }
    \setbeamertemplate{footline}{}
\begin{frame}{Triangles (2)}

\end{frame}

% Triangles (3)
\setbeamertemplate{background}{
    \begin{tikzpicture}[
        x=0.1*\the\paperheight,
        y=0.1*\the\paperheight,
        square/.style={draw,fill,minimum size=1.5pt,inner sep=0pt},
        dot/.style={circle,circle,fill,minimum size=2pt,inner sep=0pt}
        ]
        \useasboundingbox (0,0) rectangle(\the\paperwidth,\the\paperheight);
        \fill[fill=wcprimary] (0,0) rectangle(\the\paperwidth,\the\paperheight);
        \pgfmathsetseed{5984}
        \pgfmathsetmacro{\height}{sqrt(3)/2}
        % draw triangular grid
        \foreach \x in {0,...,20} {
            \foreach \y in {0,...,12} {
                \pgfmathrandominteger{\colorbox}{96}{104}
                \coordinate (A) at (\x,\y*\height);
                \coordinate (B) at (\x+1,\y*\height);
                \coordinate (C) at (\x+0.5,\y*\height+\height);
                \ifodd\x
                    \filldraw[fill=wcprimary!\colorbox,draw=wcprimary!90,opacity=0.225] (A) -- (B) -- (C) -- cycle;
                \else
                    \coordinate (D) at (\x+0.5,\y*\height-\height);
                    \filldraw[fill=wcprimary!\colorbox,draw=wcprimary!90,opacity=0.225] (A) -- (B) -- (D) -- cycle;
                \fi
            }
        }
    \end{tikzpicture}
            }
    \setbeamertemplate{footline}{}
\begin{frame}{Triangles (3)}

\end{frame}


% Squares
\setbeamertemplate{background}{
    \begin{tikzpicture}
        \useasboundingbox (0,0) rectangle(\the\paperwidth,\the\paperheight);
        \pgfmathsetseed{6523}
        \foreach \x in {1,...,20} {
            \foreach \y in {1,...,16} {
                \pgfmathrandominteger{\colorbox}{96}{104}
                \filldraw[fill=wcprimary!\colorbox,draw=wcprimary!\colorbox,line width = 0.05pt] (\x-1,\y-1) rectangle (\x,\y);
            }
        }
        \end{tikzpicture}
    }
    \setbeamertemplate{footline}{}
\begin{frame}{Squares}
\end{frame}

% Circles (1)
\setbeamertemplate{background}{
    \begin{tikzpicture}
        \useasboundingbox (0,0) rectangle(\the\paperwidth,\the\paperheight);
        \filldraw[fill=wcprimary!92,draw=wcprimary!92,line width = 0.05pt] (0,0) rectangle (\the\paperwidth,\the\paperheight);
        \pgfmathsetseed{5984}
        % Squares
        \foreach \x in {1,...,20} {
            \foreach \y in {1,...,16} {
                \pgfmathrandominteger{\colorbox}{97}{100}
                \filldraw[fill=wcprimary!\colorbox,draw=wcprimary!\colorbox,line width = 0.05pt] (\x-1,\y-1) rectangle (\x,\y);
            }
        }
        % Circles
        \foreach \i in {1,...,320} {
            \pgfmathrandominteger{\x}{1}{20}
            \pgfmathrandominteger{\y}{1}{16}
            \pgfmathsetmacro{\radiusl}{rnd*0.8}
            \pgfmathsetmacro{\radiusr}{rnd*0.8}
            \pgfmathrandominteger{\colorboxl}{98}{102}
            \pgfmathrandominteger{\colorboxr}{98}{102}
            %\filldraw[fill=wcprimary!\colorbox,draw=wcprimary!\colorbox,line width = 0.05pt] (\x-0.5,\y-0.5) circle[radius=\radius];            
            \filldraw[fill=wcprimary!\colorboxl,draw=wcprimary!\colorboxl,line width = 0.05pt] (\x-0.5,\y-0.5+\radiusl) arc[start angle=90, end angle=270, radius=\radiusl];
            \filldraw[fill=wcprimary!\colorboxr,draw=wcprimary!\colorboxr,line width = 0.05pt] (\x-0.5,\y-0.5-\radiusr) arc[start angle=-90, end angle=90, radius=\radiusr];

         }
        \end{tikzpicture}
    }
    \setbeamertemplate{footline}{}
\begin{frame}{Circles (1)}
\end{frame}

% Hexagons (1)
% Main pattern based on: https://tex.stackexchange.com/questions/579894/drawing-hexagonal-lattice-with-paths-marked-with-dark-black-and-its-midedges
% The \drawHex command creates a raised hexagon/cluster of hexagons. 
% If the coordinates are not for hexagons that touch, it will look weird.
% If you put this in your inner theme, you will need to define the command before 
% you create the \bgpattern command.
\newcommand{\drawHex}[1]{
    \path[drop shadow={opacity=0.05,shadow scale=1.02,fill=black},
            fill=wcprimary, fill opacity=1, draw=wcprimary!93, draw opacity=1]
            \foreach \coord in {#1} {
                (\coord-0) -- (\coord-1) -- (\coord-2) -- (\coord-3) -- (\coord-4) -- (\coord-5) -- cycle
            } -- cycle;
}
\setbeamertemplate{background}{
    \begin{tikzpicture}[
        x=0.1*\the\paperheight,
        y=0.1*\the\paperheight,
        square/.style={draw,fill,minimum size=1.5pt,inner sep=0pt},
        dot/.style={circle,circle,fill,minimum size=2pt,inner sep=0pt}
        ]
        \useasboundingbox (0,0) rectangle(\the\paperwidth,\the\paperheight);
        \fill[fill=wcprimary] (0,0) rectangle(\the\paperwidth,\the\paperheight);
        % draw hexagonal grid
        \foreach \x in {0,...,12} {
            \coordinate(X) at (0:1.5*\x);
            \ifodd\x
                \def\ymax{9}
                \coordinate(X) at ($(X)+(0:0.5)+(-120:1)$);
            \else
                \def\ymax{8}
            \fi
            \foreach \y in {0,...,\ymax} {
                \coordinate (\x-\y) at ($(X)+(60:\y)+(120:\y)$);
                \draw[wcprimary!90,opacity=0.225] (\x-\y) +(-60:1)
                \foreach \z [remember=\z as \lastz (initially 5)] in {0,...,5} {
                    -- coordinate(\x-\y-\lastz-m) +(\z*60:1) coordinate(\x-\y-\z)
                } -- cycle;
            } 
        }
        % Draw raised hexagons (move these around if you want to vary the pattern)
        \drawHex{8-5,8-4,7-5}
        \drawHex{9-3,9-2,8-1,9-1,10-2,10-1}
        \drawHex{1-2,2-2,2-1,2-0,3-1}
    \end{tikzpicture}
            }
    \setbeamertemplate{footline}{}
\begin{frame}{Hexagons (1)}

\end{frame}

% Hexagons (2)
\setbeamertemplate{background}{
    \begin{tikzpicture}[
        x=0.1*\the\paperheight,
        y=0.1*\the\paperheight,
        square/.style={draw,fill,minimum size=1.5pt,inner sep=0pt},
        dot/.style={circle,circle,fill,minimum size=2pt,inner sep=0pt}
        ]
        \useasboundingbox (0,0) rectangle(\the\paperwidth,\the\paperheight);
        \fill[fill=wcprimary] (0,0) rectangle(\the\paperwidth,\the\paperheight);
        \pgfmathsetseed{5984}
        % draw hexagonal grid
        \foreach \x in {0,...,12} {
            \coordinate(X) at (0:1.5*\x);
            \ifodd\x
                \def\ymax{9}
                \coordinate(X) at ($(X)+(0:0.5)+(-120:1)$);
            \else
                \def\ymax{8}
            \fi
            \foreach \y in {0,...,\ymax} {
                \pgfmathrandominteger{\colorbox}{97}{103}
                \coordinate (\x-\y) at ($(X)+(60:\y)+(120:\y)$);
                \filldraw[fill=wcprimary!\colorbox,draw=wcprimary!90,opacity=0.225] (\x-\y) +(-60:1)
                \foreach \z [remember=\z as \lastz (initially 5)] in {0,...,5} {
                    -- coordinate(\x-\y-\lastz-m) +(\z*60:1) coordinate(\x-\y-\z)
                } -- cycle;
            } 
        }

    \end{tikzpicture}
            }
    \setbeamertemplate{footline}{}
\begin{frame}{Hexagons (2)}

\end{frame}

%%%%%%%%%%%%%%%%%%%%%%%%%%%%%%%%%%%%%%%%%%%%%%%%%%%%%%%%%%%%%%%
% Other Designs
%%%%%%%%%%%%%%%%%%%%%%%%%%%%%%%%%%%%%%%%%%%%%%%%%%%%%%%%%%%%%%%

% Rose
\setbeamertemplate{background}{
    \begin{tikzpicture}
        \useasboundingbox (0,0) rectangle(\the\paperwidth,\the\paperheight);
        \fill[fill=wcprimary] (0,0) rectangle(\the\paperwidth,\the\paperheight);
        \pgfmathsetseed{652}
        % Draw r = a*sin b(theta) graph for a = 1, b = 3
        \draw[wcprimary!90,opacity=0.225,line width=1pt,shift={(0.66*\the\paperwidth,0.66*\the\paperheight)}] plot[domain=0:360,samples=360] ({\x}:{1*sin(3*\x)});
        \draw[wcprimary!90,opacity=0.225,line width=1pt,shift={(0.66*\the\paperwidth,0.66*\the\paperheight)}] plot[domain=0:360,samples=360] ({\x}:{2*sin(6*\x)});
        \draw[wcprimary!90,opacity=0.225,line width=1pt,shift={(0.66*\the\paperwidth,0.66*\the\paperheight)}] plot[domain=0:360,samples=360] ({\x}:{3*sin(12*\x)});
    \end{tikzpicture}
    }
    \setbeamertemplate{footline}{}
\begin{frame}{Rose}
\end{frame}


% Network
\setbeamertemplate{background}{
    \begin{tikzpicture}
        \useasboundingbox (0,0) rectangle(\the\paperwidth,\the\paperheight);
        \fill[fill=wcprimary] (0,0) rectangle(\the\paperwidth,\the\paperheight);
        \pgfmathsetseed{652}
        % Create a node at each x any y coordinate
        \foreach \x in {-1,...,21} {
            \foreach \y in {-1,...,17} {
                \node[draw,circle,inner sep=0pt,minimum size=1.5mm,color=wcprimary!90,opacity=0.225] (node-\x-\y) at (\x/20*\the\paperwidth,\y/16*\the\paperheight) {};
            }
        }
        % Connect each node to \x + 1 and \y+-1
        \foreach \x in {-1,...,20} {
            \foreach \y in {-1,...,16} {
                \pgfmathtruncatemacro{\xnext}{\x+1}
                \pgfmathtruncatemacro{\ynext}{\y+1}
                \pgfmathtruncatemacro{\ylast}{\y-1}
                \draw[wcprimary!90,opacity=0.225] (node-\x-\y) -- (node-\xnext-\y);
                \draw[wcprimary!90,opacity=0.225] (node-\x-\y) -- (node-\xnext-\ynext);
                \ifnum \y>0
                    \draw[wcprimary!90,opacity=0.225] (node-\x-\y) -- (node-\xnext-\ylast);
                \fi
            }
        }
    \end{tikzpicture}
    }
    \setbeamertemplate{footline}{}
\begin{frame}{Network}
\end{frame}

% Quilt
\setbeamertemplate{background}{
    \begin{tikzpicture}
        \useasboundingbox (0,0) rectangle(\the\paperwidth,\the\paperheight);
        \fill[fill=wcprimary] (0,0) rectangle(\the\paperwidth,\the\paperheight);
        \foreach \i in {1,...,20}
            \foreach \j in {1,...,14}
                \node[draw,circle,minimum size=30mm,color=wcprimary120,opacity=0.3] at ({(\i-2)*(\the\paperwidth/16)},{(\j-2)*(\the\paperheight/10)}) {};
    \end{tikzpicture}
    }
    \setbeamertemplate{footline}{}
\begin{frame}{Quilt}
\end{frame}

% Circles (2)
\setbeamertemplate{background}{
    \begin{tikzpicture}
        \useasboundingbox (0,0) rectangle(\the\paperwidth,\the\paperheight);
        \fill[fill=wcprimary130] (0,0) rectangle(\the\paperwidth,\the\paperheight);
        \node[draw,circle,minimum size=\the\paperwidth,fill=wcprimary120,color=wcprimary120,line width=0.05pt] at (0.5*\the\paperwidth,0.5*\the\paperheight) {};
        \node[draw,circle,minimum size=\the\paperwidth,fill=wcprimary110,color=wcprimary110,line width=0.05pt] at (0.25*\the\paperwidth,0.5*\the\paperheight) {};
        \node[draw,circle,minimum size=\the\paperwidth,fill=wcprimary,color=wcprimary,line width=0.05pt] at (0,0.5*\the\paperheight) {};
    \end{tikzpicture}
    }
    \setbeamertemplate{footline}{}
\begin{frame}{Circles (2)}

\end{frame}

% Circles (3)
\setbeamertemplate{background}{
    \begin{tikzpicture}
        \useasboundingbox (0,0) rectangle(\the\paperwidth,\the\paperheight);
        \fill[fill=wcprimary] (0,0) rectangle(\the\paperwidth,\the\paperheight);
        \node[draw,circle,minimum size=\the\paperwidth,color=wcprimary90,line width=1pt] at (0.2*\the\paperwidth,0.5*\the\paperheight) {};
        \node[draw,circle,minimum size=\the\paperwidth,color=wcprimary40,line width=2pt] at (0.27*\the\paperwidth,0.5*\the\paperheight) {};
        \node[draw,circle,minimum size=\the\paperwidth,color=wcprimary70,line width=4pt] at (0.25*\the\paperwidth,0.65*\the\paperheight) {};
    \end{tikzpicture}
    }
    \setbeamertemplate{footline}{}
\begin{frame}{Circles (3)}

\end{frame}




\end{document}